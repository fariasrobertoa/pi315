\section{Introducción}
\label{sec:intro}

La evaluación de usabilidad~\cite{ISO1998} automática refiere a la posibilidad de disponer de medios basados en software para realizar alguna de las actividades en las que se descompone (captura de datos, análisis o crítica)~\cite{IH2001}. En lo que respecta a la captura de datos automatizada se ha caracterizado por usar archivos de logs. Sus ventajas son varias: soportan la ejecución de las pruebas en laboratorios de usabilidad o en el ambiente natural del usuario; son transparentes para los usuarios; permiten con facilidad obtener información de múltiples usuarios; permiten detectar errores comunes y actuales; etc. Los logs aportan gran cantidad de datos a los evaluadores de usabilidad, además de posibilitar la contrastación del comportamiento entre diferentes usuarios. Sin embargo, tienen desventajas, los logs generados por servidores webs o sistemas operativos, generan una gran cantidad de datos que resulta irrelevante para la evaluación de usabilidad, y son incapaces de registrar información específica. Por otro lado, añadir en las aplicaciones el código que realiza el logging de las actividades y  tareas de usuario que interesan al evaluador de usabilidad, implica invadir la estructura interna de las aplicaciones agregando cientos de líneas de código, cuya remoción posterior es muy compleja.

Las técnicas de separación de concerns~\cite{KLM+1997} \colorbox{green}{[3]}, y en particular de la Programación Orientada a Aspectos (AOP)~\cite{NIELSEN1992} \colorbox{green}{[4]}, proporcionan mecanismos para diseñar e implementar módulos de código que realicen la traza y el logging sin invadir y/o alterar la estructura interna de los sistemas. La posibilidad de realizar la recolección de los datos en forma dinámica, transparente para el usuario y separada del código de las aplicaciones a evaluar, ha provocado interés y varios enfoques se han presentado que aplican AOP a la evaluación de usabilidad. En particular aquellos propuestos para aplicaciones de escritorio (WIMP) ofrecen esquemas para la evaluación a bajo nivel. Posibilitan la traza de los elementos o eventos de interfaces y/o errores pero sin un contexto de significado (tarea) para el evaluador. En consecuencia carecen de una relevancia real a la hora de interpretar los resultados y/o calcular métricas, y proporcionar información útil para emitir una opinión al evaluador de usabilidad. 

Luego de analizar los diseños presentados, se puede concluir que la razón radica en el hecho de que el concepto de “tarea” no es común en los vocabularios que participan de un escenario de evaluación de usabilidad y por ende no está presente en ambos contextos. En los sistemas software se representan los conceptos del dominio y lógica de negocios tales como: Cuenta, Cliente, Factura, Impuesto, Stock, etc., mientras que los evaluadores de usabilidad buscan identificar, analizar e interpretar  tareas de usuario. En los sistemas software  las “tareas de usuario” resultan ser entidades intangibles e inexistentes. Es por ello que los módulos (aspectos) que se diseñan e implementan para llevar adelante en forma automática las pruebas de la usabilidad, se restringen a los conceptos (entidades) existentes (métodos / atributos y clases del dominio), perdiéndose la concepción de la tarea. Planteamos como hipótesis, que sí es posible colocar en el centro la tarea, como concepto y como entidad, y por medio de aspectos conectarla al dominio y a los servicios que se requieran para la evaluación, entonces podrá ser identificada y analizada. En consecuencia se obtendrán datos que puedan ser interpretados a mayor nivel semántico y se lograrán resultados de mayor nivel de abstracción. 

En este trabajo presentamos un framework que permite realizar la evaluación de la usabilidad en aplicaciones de escritorio en el contexto de tareas de usuario. El framework consiste de un conjunto de módulos, aspectos y clases, que interactúan y se relación de manera tal de ser altamente reusable y sus requerimientos de especialización (configuración) son mínimos. 
Este trabajo se estructura de la siguiente manera, en la Sección~\ref{sec:eval_usabilidad} se describe brevemente el modelo de usabilidad soportado por el framework, en la Sección~\ref{sec:framework_aspectual} se presenta el Framework, en la Sección~\ref{sec:casos_de_estudio} se exponen algunos casos de estudio, en la Sección~\ref{sec:trabajos_relacionados} se plantean los trabajos relacionados y finalmente las conclusiones y trabajo futuro. 
