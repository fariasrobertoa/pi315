\section{Conclusiones}
\label{sec:conclusiones}
En este trabajo se ha presentado un framework que permite la evaluación de la usabilidad de aplicaciones de escritorio de alto nivel, dado que permite obtener información y calcular métricas a nivel de tareas de usuario, en las dimensiones de eficiencia, efectividad y satisfacción. El framework automatiza la recolección de datos y algunas funciones de análisis. Aunque el framework se compone de varios módulos, el desarrollador sólo debe configurar dos pointcut de un solo aspecto (TaskConnect), para que toda la evaluación de usabilidad de una determinada tarea sea efectuada. Lo cual implica un alto nivel de reúso, del diseño y la implementación, ya que se ha logrado reducir al mínimo las condiciones de variación del problema y requerimientos de configuración. 
Los trabajos futuros refieren a incorporar más funcionalidad para la registración de otros datos relacionados a la tarea con el objeto de calcular nuevas métricas. También resulta de utilidad incorporar un visor que facilite la tarea de análisis y crítica posterior al evaluador de usabilidad. Completadas estas mejoras, el paso siguiente es desarrollar y evaluar un framework para aplicaciones web.
