\section{Trabajos Relacionados}
\label{sec:trabajos_relacionados}
Como se mencionó anteriormente existen contribuciones que aplican AOP para la evaluación de la usabilidad de aplicaciones de escritorio. La principal diferencia con estos trabajos es que no distinguen a la Tarea entre los aspectos. Los aspectos hacen el seguimiento de eventos y registran datos simplemente. Se requiere mayor intervención del desarrollador, es decir, son esquemas menos reusables y flexibles.
Tarta y Moldovan~\cite{TM2006} [7] proponen un diseño de aspectos que soporta la evaluación automática de la usabilidad en aplicaciones de escritorio. Se plantea que mediante una jerarquía de aspectos se pueden reusar pointcuts que compartan los mismos puntos de entrada y salida (join-points). Los aspectos derivados definen nuevos pointcuts para la intercepción de errores, completitud de tareas, captura de pantallas y adquisición de datos por medio de cuestionarios. Los aspectos concretos por medio de los advices realizan el log de los datos. Aquí los aspectos hacen el seguimiento y el logging en el mismo aspecto pero en reiteradas veces dado que existen aspectos para los errores, el cálculo del tiempo, etc. y por ende el desarrollador debe configurar mas aspectos, pero se pierde la noción de la tarea, por lo que toda la estructura es específica de cada aplicación en la que se use. Se presentan simples diagramas y código de ejemplo en AspectJ.
Shekh y Tyerman~\cite{ST2010} [8] presentan el desarrollo de un framework AOP para la evaluación de la usabilidad con AOP. Las autoras registran los eventos de IU, por ejemplo los eventos del mouse. Respecto de las características del framework y diseño de aspectos no se brindan detalles, pero se indica que han seguido las recomendaciones de~\cite{TM2006} [7]. El framework esta desarrollado en AspectJ. Se presentan resultados de experimentos controlados y pautados en un laboratorio.

Holzinger y otros~\cite{HBS2011} [9] proponen la implementación de aspectos para seguir la trayectoria de algunos eventos de interface (entrada del teclado, menúes y acciones arrastrar y soltar). Se plantea usar Objective-C, agregando los aspectos a la jerarquía de objetos usando la tecnología “método swizzling”, y extendiendo de clases. El diseño se centra en un aspecto que realiza el logging de manera centralizada y tres aspectos son necesarios para identificar los eventos de interface mencionados. 
\colorbox{green}{Yonglei Tao}~\cite{TAO2008} [10] propone AOP para capturar automáticamente eventos de la interface de usuario en aplicaciones con arquitectura MVC (modelo-vista-controlador). Se propone una jerarquía de aspectos, en la cual se logra reusar solo el método que realiza el reporte (hora-fecha y evento ocurrido), los pointcuts y advices deben ser redefinidos en cada caso y aplicación, por ejemplo actualización de observer, manejadores y notificadores de eventos, cuadros de diálogo. Se presentan códigos de ejemplo en AspectJ y un caso de estudio simple.  
Un esquema muy similar al anterior, es el que presenta \colorbox{green}{Yonglei Tao}~\cite{TAO2012} [11], ya que este enfoque se basa en técnicas OA para ejecutar las trazas de los eventos de interface de usuario y recolectar información contextual, en aplicaciones de escritorio (WIMP). Propone una jerarquía de aspectos, la cual consiste básicamente en un método  que contiene la lógica que permite hacer el reporte, los sub-aspectos definen los eventos a interceptar en los pointcuts y los advice que invocan al método que ejecuta el reporte. Se presentan simples códigos de ejemplo en AspectJ.
Bateman y otros~\cite{BGO+2009} [12] presentan el enfoque “Instrumentación Interactiva de Usabilidad” (IIU). Los evaluadores de usabilidad especifican qué acciones serán registradas (log) interactuando con los elementos de la interfaz de la aplicación objeto de la evaluación, eliminando la necesidad de apoyo adicional de programación. Esta primera actividad se apoya en AOP, y permite la instrumentación a través de la interacción directa con la aplicación a ser instrumentada; se decide qué elementos (interfaz) de un sistema están relacionados con tareas particulares y problemas de usabilidad. 
Humayoun y otros~\cite{HDC2009} [13] presentan la herramienta UEMan que permite gestionar y automatizar actividades de la evaluación de la usabilidad desde un enfoque UCD (diseño centrado en el usuario) durante el proceso de desarrollo de software. El modelo y la herramienta incorporan el concepto de tarea de usuario y facilita la ejecución de diversos tipos de experimentos: experimentos de evaluación heurística, experimentos de tipo tarea, experimentos basados en cuestionarios y experimentos dinámicos que emplean logging implementado con aspectos. La herramienta usa aspectos para operaciones muy básicas como, hacer el Timer (duración) de una actividad (desde un punto de entrada hasta un punto de salida, definido mediante dos pointcuts) y contar la cantidad de clicks del mouse y teclas presionadas en dicho intervalo.
