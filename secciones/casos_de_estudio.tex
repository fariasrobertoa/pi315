\section{Casos de Estudio}
\label{sec:casos_de_estudio}
 
 El desarrollo iterativo del framework implicó la ejecución de pruebas con cada versión del framework obtenida. El mismo ha sido probado y validado empleando dos aplicaciones de escritorio JMoney (http://jmoney.sourceforge.net/) y Freemind (http://freemind.sourceforge.net/). 
 
\begin{description}

\item[JMoney]

Es un gestor de finanzas de uso personal. Tiene ??? paquetes, 83 clases, 594 métodos y 436 atributos. Se compone de 5780 líneas de código. 

\item[Freemind]

Es una herramienta para la creación de mapas mentales. Tiene 50 paquetes, 820 clases, 6974 métodos, 2816 atributos. Se compone de 108.378 líneas de código. 
\end{description}

Para cada una se crearon diversas tareas y en cada iteración se configuró el framework. Los principales objetivos de las pruebas fueron constatar que el registró de los datos y calculo de las métricas funinaba correctamente; y comprobar el nivel de reúso y configuración y especialización. A continuación se presenta una tarea para cada caso de estudio, el aspecto que fue adaptado  y el resultado de las pruebas.

Freemind
Tarea: Creación de un Mapa Mental Básico.
1) Crear un nuevo mapa, haciendo clic en el botón "Nuevo" desde la barra de herramientas, o desde el menú Archivo. 
2) Completar el texto del nodo raíz del mapa creado.
3) Construir una jerarquía de tres niveles con nodos hijos y hermanos (al menos 11).Para ello se deberá posicionar en el nodo sobre el cual quiere crear un nodo hijo o hermano y luego insertar el nodo desde las opciones del menú contextual, o desde el menú Insertar.
4) Ir al menú herramientas y ordenar los nodos por nombre.
5) Guardar el mapa mental en el escritorio con un nombre significativo.
Inicio de la tarea: La tarea se considerará iniciada luego que el usuario haya seleccionado la opción "Nuevo Mapa" del menú principal o desde la barra de herramientas.
Fin de la tarea: La tarea se considerará finalizada luego que el usuario guarda, por primera vez, el mapa en un archivo con el nombre solicitado. 

\begin{verbatim}
public aspect FreemindTask1extends TaskConnect{
     private String idTask=”Creación de un Mapa Conceptual Básico”;
     pointcut startTask():execution(void freemind.modes.common.actions.NewMapAction.actionPerformed(ActionEvent));
     pointcut endTask():execution(* freemind.modes.ControllerAdapter.SaveAsAction.*(..))||execution(* freemind.modes.ControllerAdapter.SaveAction.*(..));
}

\end{verbatim}

%insertar fig4.png

JMoney
Tarea: Crear una nueva cuenta
1) Hacer click en el botón derecho del panel lateral y seleccionar la opción "new account".
2) Ingresar las propiedades de la cuenta desde el panel properties de la cuenta
3) Ingresar entradas en la cuenta desde el panel entries  
4) Salvar desde el botón "save" en la barra de herramientas o desde menú "file/save", usando combinación de teclas "ctrl+s" o respondiendo afirmativamente al mensaje de grabar los cambios antes de cerrar la aplicación.

\begin{verbatim}
public aspect Tarea1JMoneyConnect extends TareaConnect{		
     private String idTask=”Crear nueva cuenta”;
     pointcut starTask():execution(void   
                  net.sf.jmoney.gui.MainFrame.newAccount());
     pointcut endTask():execution(void                  
               net.sf.jmoney.gui.MainFrame.saveSession()) || 
       execution(void net.sf.jmoney.gui.MainFrame.saveSessionAs());
}

\end{verbatim}

%insertar fig5.png
